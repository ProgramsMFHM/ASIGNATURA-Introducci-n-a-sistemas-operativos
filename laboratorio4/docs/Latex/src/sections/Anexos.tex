\section{Anexos}

\begin{lstlisting}[style=CodeStyle, caption={Pseudocodigo del problema de los filósofos comensales}, label={lst:Pseudocodigo}]
// Definiciones iniciales

#define N 5 /* Numero de filosofos */
#define LEFT (i-1)%N /* Vecino a la izquierda de i */
#define RIGHT (i+1)%N /* Vecino a la derecha de i */
#define THINKING 0 /* Filosofando */
#define HUNGRY 1 /* Intentando comer */
#define EATING 2 /* Comiendo */

// Variables a utilizar
typedef int semaphore; /* Semaforos como int */
int state[N]; /* Estado de los filosofos */
semaphore mutex = 1; /* Exclusion mutua */
semaphore s[N]; /* Un semaforo por filosofo */

// Funcion que ejecutaria el i-esimo filosofo
philosopher(i)
int i;
{
    while (TRUE) {
    think();
    take_forks(i); /* Toma dos tenedores o bloquea */
    eat();
    put_forks(i); /* Deja los tenedores */
    }
}

// Funcion para que un filosofo tome sus tenedores
take_forks(i)
int i;
{
    down(mutex);
    state[i] = HUNGRY;
    test(i);
    up(mutex);
    down(s[i]);
}

// Funcion para revisar si filosofo i tiene la posibilidad de comer
test(i)
int i;
{
    if (state[i] == HUNGRY && state[LEFT] != EATING && state[RIGHT] != EATING) {
    state[i] = EATING;
    up(s[i]);
    }
}

// Funcion para devolver los tenedores del filosofo i a su lugar
put_forks(i)
int i;
{
    down(mutex);
    state[i] = THINKING;
    test(LEFT);
    test(RIGHT);
    up(mutex);
}
\end{lstlisting}