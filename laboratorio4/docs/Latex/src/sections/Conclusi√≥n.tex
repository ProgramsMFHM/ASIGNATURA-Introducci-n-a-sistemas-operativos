\section{Conclusiones}
Como se observó a lo largo del escrito, el problema de los filósofos es un reto que puede ser abarcado desde distintas aristas, algunas más eficientes o completas que otras, pero todas con un mismo resultado, la correcta sincronización de los filósofos (de los procesos).

En este caso, se presentó una única solución, la cual se puede considerar como completa, debido a que se abarca el problema de manera general y se resuelven los problemas presentados en el enunciado entregado en la sección \ref{sec:Introduccion}.

Es destacable mencionar el cómo este problema y su solución se pueden extrapolar en la modelación aplicaciones reales, como los son los problemas de E/S en los sistemas operativos.

Finalmente, se puede decir con seguridad que el objetivo planteado en la sección \ref{sec:Objetivos} fue cumplido a cabalidad habiendo comprendido el problema de los filósofos y una solución eficiente mediante el uso de hebras y semáforos.





