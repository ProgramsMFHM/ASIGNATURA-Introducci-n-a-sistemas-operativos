\section{Procedimiento experimental y Análisis }\label{sec:ProcedimientoExperimental}
Para la realización de este experimento se utilizó como base el pseudocódigo entregado por el profesor, el cual se puede encontrar en \ref{lst:Pseudocodigo}. Con base en este, se realizó un programa en C, el cual utiliza las bibliotecas:
\begin{itemize}
    \item \texttt{semaphore.h}, encargada de proporcionar las funciones y estructuras necesarias para manejar semáforos.
    \item \texttt{pthread.h}, encargada de proporcionar las funciones y estructuras necesarias para manejar hilos.
    \item Demás bibliotecas estándar en el lenguaje C para entrada y salida, manejo de números aleatorios, etc$\cdots$
\end{itemize}

El programa realizado se encarga de inicializar a los ``Filósofos'' mediante el uso de un hilo para cada uno, convirtiendo a cada uno de estos en un proceso liviano encargado de realizar las acciones de pensar, comer y tomar los tenedores. Para la coordinación entre estos, se utilizan semáforos los cuales controlan el acceso a los tenedores, de tal manera que los filósofos no puedan tomar los tenedores al mismo tiempo.

La ejecución del programa se realizó mediante impresiones por pantalla, las cuales van representando y detallando el estado en el cual se encuentra cada filósofo [THINKING, EATING, HUNGRY]. Además, se utilizó la función \texttt{sleep()} para simular el tiempo que tarda un filósofo en realizar cada acción (entre $1$ y $10$ segundos).

\subsection{Afrontando el problema}
\subsubsection*{Los tenedores han desaparecido}
Para la solución del anterior problema no se tuvo en cuenta como tal la existencia de los tenedores, sino que se planteó el problema de la siguiente manera: ``Un filósofo no puede empezar a comer si alguno de los filósofos adyacentes está comiendo'', este enunciado, es equivalente al anteriormente mencionado pero permite una solución enfocada de otra manera.

En particular esto se puede observar en la función \texttt{test()} del programa, la cual verifica si un filósofo está intentando comer, y si aquellos adyacentes están comiendo, en caso de que no lo estén el filósofo cambiara su estado a EATING y puede proceder a comer.

\begin{lstlisting}[style=CodeStyle, caption={Función para revisar si filosofo i tiene la posibilidad de comer}, label={lst:FuncionTest}, language=C]
void test(int ID){
    // En caso de que tenga hambre compruebo si puedo comer.
    if(ProgramInfo.state[ID] == HUNGRY){
        if(ProgramInfo.state[LEFT] != EATING &&
            ProgramInfo.state[RIGHT] != EATING
        ){
            // En caso de poder entonces indico que estoy comiendo
            ProgramInfo.state[ID] = EATING;
            // En caso de que estuviera bloqueado por esperar a que los demas terminen de comer entonces me desbloqueo
            // Si no estoy bloqueado esta senhal evita que me bloquee por accion de la funcion "take\_forks()"
            sem_post(&ProgramInfo.s[ID]);
        }
        else{
            printf("Filosofo %d" ANSI_COLOR_RED " ESPERANDO" ANSI_COLOR_RESET "\n", ID);
        }
    }
}
\end{lstlisting}

\subsubsection*{Manejo de los semáforos}
El semáforo más utilizado en este programa es el \texttt{mutex}, que evita que una hebra cambie el estado del filósofo correspondiente otra está en este mismo asunto, lo que evita que ``los filósofos se confundan'' o ``realicen acciones simultáneas'' lo que podría llevar a errores en el código.

Adicionalmente cada filósofo tiene su propio semáforo que permite controlar si está o no esperando a que el resto de los filósofos terminen de comer. Cuando un filósofo decide comer primero usa la función \texttt{test()} que, en caso de ser exitosa, llama a la función \texttt{sem\_post()} (UP), sin embargo en \textbf{cualquier caso} luego de hacer el test se llama a la función \texttt{sem\_wait()} (DOWN) lo que tendrá una de dos consecuencias:
\begin{enumerate}
    \item Si el filósofo logró ``tomar los tenedores'' en la función test(), entonces su semáforo tendrá el valor de $1$ por lo que el \texttt{sem\_wait()} \textbf{no bloqueará} el proceso, sino que este se seguirá ejecutando (procediendo a comer).
    \item Si el filósofo no logró ``tomar los tenedores'', entonces su semáforo se mantendrá en $0$ y el \texttt{sem\_wait()} bloqueará el proceso hasta que ambos filósofos adyacentes no estén comiendo (Esto se maneja ya que cuando un filósofo termina de comer manda una señal a los adyacentes para usar la función \texttt{test()}, que provocará el desbloqueo \textbf{en caso de que sea necesario}).
\end{enumerate}

\begin{lstlisting}[style=CodeStyle, caption={Función para que un filosofo tome sus tenedores}, label={lst:FuncionTakeForks}, language=C]
void take_forks(int ID){
    sem_wait(&ProgramInfo.mutex);       // Entro en la seccion critica
    ProgramInfo.state[ID] = HUNGRY;     // Indico que tengo hambre
    test(ID);                           // Compruebo si puedo o no comer
    sem_post(&ProgramInfo.mutex);       // Salgo de la seccion critica
    sem_wait(&ProgramInfo.s[ID]);       // Me bloqueo en caso que no haya podido tomar los tenedores
}
\end{lstlisting}