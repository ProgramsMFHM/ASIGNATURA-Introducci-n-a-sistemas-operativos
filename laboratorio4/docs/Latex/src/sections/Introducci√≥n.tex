\section{Introducción} \label{sec:Introduccion}
``Cinco filósofos se sientan en una mesa redonda. Cada uno cuenta con un plato de espagueti y un tenedor a su lado. El espagueti es tan escurridizo que un filósofo necesita de DOS tenedores para comerlo. Entre cada dos platos existe un tenedor, por lo que existe un número de filósofos igual al de tenedores. Cada filósofo cuenta con dos períodos de tiempo: uno en el que come y otro en el que piensa. Al sentir hambre, cada filósofo intenta tomar los cubiertos a sus lados, lo intenta con la izquierda y lo intenta con la derecha, toma un bocado del espagueti y lo come. Al terminar, deja los tenedores en su lugar y vuelve a pensar.'' El problema propuesto anteriormente es el llamado: ``\textit{Problema de los filósofos comensales}'', este problema fue propuesto por \textbf{Edsger W. Dijkstra}, e ilustra un problema de \textbf{sincronización de procesos}.
Con esto en mente, el informe actual se encargará de abordar el problema, basándose principalmente en el pseudocódigo \footnote{Un pseudocódigo es una manera de describir la lógica de un algoritmo o programa en lenguaje natural, imitando la estructura de un código sin seguir las reglas de un lenguaje de programación.} propuesto por el académico (estrácto de código \ref{lst:Pseudocodigo}) e implementar una solución en lenguaje C, analizando el comportamiento de la solución propuesta.

\begin{figure}[!ht]
    \centering
    \includegraphics[width=0.4\textwidth]{src/images/Filósofos.png}
    \caption{Problema de los filósofos comensales}
    \label{fig:Problema_de_los_filosofos}
\end{figure}