\begin{abstract}
Un proceso que se ejecute aislado de los demás en un sistema operativo puede ser útil en diversas ocasiones donde no se precise de información proveniente ``del exterior''; sin embargo, en muchas ocasiones es necesario que el proceso tenga acceso a información proveniente de otros procesos o sistemas. Para esto se utiliza la \textbf{comunicación entre procesos}, que consiste en la transferencia de mensajes entre procesos. En el presente informe abordaremos una de las opciones de comunicación entre procesos, las colas de comunicación, lo anterior mediante el lenguaje de programación C y sus bibliotecas \texttt{sys/ipc.h} y \texttt{sys/msg.h}.
\end{abstract}