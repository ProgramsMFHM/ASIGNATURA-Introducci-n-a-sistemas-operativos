\section{Introducción}
Habiendo comprendido ya la jerarquía entre procesos mediante el uso de funciones como \texttt{fork()}, el siguiente paso para el manejo correcto de procesos dentro de un sistema operativo consiste en comprender la comunicación entre dos procesos.

Como se demostró en el \href{https://github.com/ProgramsMFHM/ASIGNATURA-Introduccion-sistemas-operativos/tree/main/laboratorio2}{anterior experimento de laboratorio} los procesos creados mediante la función \texttt{fork()} no comparten memoria con el padre que los creó; lo anterior dificulta sobremanera el uso de este método para crear procesos que aprovechen la capacidad de computo del sistema operativo.

Entonces, ¿Existe un método para permitir que dos procesos compartan información entre sí de manera efectiva? Lo cierto es que existen muchas opciones para comunicar procesos entre sí, pero en el presente informe veremos una de ellas: la \textbf{Transferencia de mensajes}: ``\textit{Este método de comunicación entre procesos utiliza dos primitivas SEND y RECEIVE (...) son llamadas al sistema y no construcciones del lenguaje. Como tales, es fácil colocarlas en procedimientos de biblioteca, como \texttt{send(destino, \&mensaje);} y \texttt{receive(origen, \&mensaje);}.
La primera llamada envía un mensaje a un destino dado, y la segunda recibe un mensaje de un origen dado (o de cualquiera [ANY] si al receptor no le importa). Si no hay un mensaje disponible, el receptor podría bloquearse hasta que uno llegue. Como alternativa, podría regresar de inmediato con un código de error.}'' \parencite{tanenbaum1997sistemas}