\section{Procedimiento experimental}
El Procedimiento experimental propuesto para el presente laboratorio es el siguiente:

\begin{enumerate}
    \item Crear tres procesos generadores de palabras.
    \begin{itemize}
        \item Primer proceso Genera números enteros aleatorios entre 0 y 100.
        \item Segundo proceso Genera letras aleatorias de la A a la Z.
        \item Tercer proceso Genera una letra acompañada de un numero entero.
    \end{itemize}
    \item Para cada generador se debe considerar que tras cada creación debe haber un tiempo de retardo delay() entre 0 y 3 seg
    \item Determinar el tiempo de ejecución de cada generador de palabras.
    \item Correr los tres generadores al mismo tiempo en forma desatendida, liberando así  la consola de comandos.
    \item Guardar los datos generados en tiempo de ejecución de cada uno de los tres generadores.
    \item Si el generador es infinito, ver; Número de proceso en ejecución y eliminación del proceso.
\end{enumerate}

Mediante la realización del presente informe buscaremos responder las siguientes preguntas relacionadas con los procesos:
\begin{enumerate}
    \item De que manera se puede realizar verdadera aleatoriedad de un numero?
    \item Si los tres generadores crean 10.000 palabras cada uno, demoran el mismo tiempo de ejecución?
    \item Si se crean 50 generadores idénticos, el tiempo de ejecución es el mismo para todos?
    \item De que manera se pueden eliminar todos los procesos en ejecución de una sola vez
    \item Cuál es el comando que elimina procesos?
    \item Si quitemos los delay de tiempos, la ejecución de los generadores será la misma?  si no es así, a que se debe?
    \item Qué otros procesos ejecuta el Sistema Operativo mientras se ejecutan los que que generamos?
\end{enumerate}
