\section{Conclusiones y recomendaciones}
Finalmente a raíz de lo realizado en este experimento se puede concluir lo siguiente:
\begin{enumerate}
  \item Se pudo comprender de manera teórica el funcionamiento de los procesos concurrentes dentro de un Sistema Operativo.
  \item Además se logró llevar a cabo la creación de tres procesos concurrentes que generan números aleatorios, letras aleatorias y letras con números, registrando sus resultados en archivos de manera `Simultánea'.
  \item Se consiguió comprender el funcionamiento de la generación de números aleatorios dentro de sistemas computacionales, comprendiendo la naturaleza no aleatoria de los mismos y los métodos de generación de números pseudoaleatorios usados en los sistemas actuales, en particular el lenguaje C.
\end{enumerate}

Los objetivos de este laboratorio se cumplieron a cabalad, sin embargo quedan pendientes diversas mejoras que se pueden realizar, como por ejemplo:
\begin{enumerate}
  \item Comprender el funcionamiento de procesos padres e hijos permitiendo el manejo de procesos concurrentes de mejor manera.
  \item Comprender el funcionamiento de la comunicación entre procesos, permitiendo enviar mensajes (más allá de lo aprendido con `killall') y manejar los mensajes recibidos de manera adecuada.
\end{enumerate}