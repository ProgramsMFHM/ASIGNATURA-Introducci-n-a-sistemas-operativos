\begin{abstract}
    La gestión y manejo de procesos dentro de un sistema operativo corresponden     a una parte de vital importancia dentro de este. Cuando un proceso esta en ejecución este es capaz de duplicarse y crear otros procesos `hijos' lo que permite alivianar la carga de un proceso y aprovechar de mejor manera el `pseudoparalelismo' de los sistemas operativos. De esto surge la pregunta, \textbf{¿Es posible beneficiarse de esta capacidad para la resolución de problemas asociados a los procesos?}, pregunta a la que se le dará respuesta en este escrito.
\end{abstract}