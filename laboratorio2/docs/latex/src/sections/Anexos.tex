\section{Anexos}

\lstinputlisting[language=C, style=CodeStyle, caption={Hijo Padre Abuelo}, label={lst:arbol_gen}]{../../src/programa1.c}


\lstinputlisting[language=C, style=CodeStyle, caption={Uso de hijos en C Millton}, label={lst:uso_de_hijos_milton}]{../../src/programa2.c}

\begin{lstlisting}[language=C, style=CodeStyle, caption={Uso de hijos en C por Ayrton}, label={lst:uso_de_hijos_ayrton}]
/// @author Ayrton Morrison R.
/// @brief Generacion de procesos Padre, hijo, nieto para operaciones aritmeticas.

#include <stdio.h>
#include <stdlib.h>
#include <math.h>
#include <unistd.h>

int main(int argc, char **argv){
    if(argc != 4){
        printf("Error, ingrese 3 numeros\n");
        return 0;
    }
    float firstNumber = atof(argv[1]);
    float secondNumber = atof(argv[2]);
    float thirdNumber = atof(argv[3]);
    if(!fork()){
        // PROCESO HIJO (SEGUNDO PROCESO)
        printf("Soy el proceso hijo y duplicare %.1f\n", firstNumber);
        printf("El resultado es %.2f\n", firstNumber * 2);
        if(!fork()){
            // PROCESO NIETO (ULTIMO PROCESO)
            printf("Soy el proceso nieto y sacare la raiz de %.1f\n", thirdNumber);
            if(thirdNumber < 0){
                printf("Error, no se puede sacar la raiz de un numero negativo\n");
                return 0;
            }
            printf("El resultado es %.2f\n", sqrt(thirdNumber));
        }
    }
    else{
        // PROCESO PADRE (PRIMER PROCESO)
        printf("Soy el proceso padre y elevare a la potencia de 3 %.1f\n", secondNumber);
        printf("El resultado es %.2f\n", pow(secondNumber, 3));
    }
    return 0;
}

\end{lstlisting}

\begin{lstlisting}[language=C, style=CodeStyle, caption={Uso de hijos en C para una ecuacion cuadratica, por Ayrton}, label={lst:cuadratica_ayrton}]
    /// @author Ayrton Morrison R.
    /// @brief Generacion de procesos Padre e hijos para la resolucion de una ecuacion cuadratica.

#include <stdio.h>
#include <stdlib.h>
#include <math.h>
#include <unistd.h>
#include<sys/wait.h>
#include<sys/types.h>

typedef struct imaginary{
    float real;
    float imaginary;
}imaginary;
float getDiscriminant(float a, float b, float c);

int main(int argc, char **argv){
    if(argc > 4 || argc < 4){
        printf("Error, ingrese 3 numeros en el formato 'a b c', siendo estos coeficientes de una ecuacion cuadratica \n");
        return 0;
    }
    float a = atof(argv[1]);
    float b = atof(argv[2]);
    float c = atof(argv[3]);
    float part_1 = 0.0, part_2 = 0.0, part_3 = 0.0;
    pid_t pid1, pid2;
    if(a == 0){
        printf("Error, el coeficiente 'a' no puede ser 0\n");
        return 0;
    }
    pid1 = fork();
    if(!pid1){
        //PROCESO HIJO(SEGUNDO PROCESO)
        part_1 = 2 * a;
        FILE *file = fopen("answers.txt", "w");
        fprintf(file, "%.3f\n", part_1);
        fclose(file);
        pid2 = fork();
        if(!pid2){
            //PROCESO NIETO(ULTIMO PROCESO)
            part_2 = -b;
            FILE *file = fopen("answers.txt", "a");
            fprintf(file, "%.3f", part_2);
            fclose(file);
            exit(0);
        }
        else{
            wait(NULL);
            exit(0);
        }
    }
    else{
        //PROCESO PADRE(PRIMER PROCESO)
        part_3 = getDiscriminant(a, b, c);
        wait(NULL);
        wait(NULL);
        FILE *file = fopen("answers.txt", "r");
        fscanf(file, "%f\n", &part_1);
        fscanf(file, "%f\n", &part_2);
        fclose(file);
        //2 RESPUESTAS DISTINTAS Y REALES
        if (part_3 > 0){
            float x1 = (part_2 + sqrt(part_3)) / part_1;
            float x2 = (part_2 - sqrt(part_3)) / part_1;
            printf("Las respuestas son %.3f y %.3f\n", x1, x2);
        }
        // 2 RESPUESTAS IGUALES Y REALES
        else if(part_3 == 0){
            float x = part_2 / part_1;
            printf("La respuesta es %.3f\n", x);
        }
        // 2 RESPUESTAS DISTINTAS Y COMPLEJAS
        else if(part_3 < 0){
            imaginary x1, x2;
            x1.real = part_2 / part_1;
            x1.imaginary = sqrt(-part_3) / part_1;
            x2.real = part_2 / part_1;
            x2.imaginary = -sqrt(-part_3) / part_1;
            printf("Las respuestas son %.3f + %.3fi y %.3f - %.3fi\n", x1.real, x1.imaginary, x2.real, x2.imaginary);
        }
    }
}
float getDiscriminant(float a, float b, float c){
    return ( pow(b,2) - (4*a*c) );
}
\end{lstlisting}

\lstinputlisting[language=C, style=CodeStyle, caption={TAD para números complejos, por Milton}, label={lst:coplex.h}]{../../incs/complex.h}

\lstinputlisting[language=C, style=CodeStyle, caption={Ecuación cuadrática en forma secuencial, por Milton}, label={lst:cuadratic1.c}]{../../src/cuadratic1.c}

\lstinputlisting[language=C, style=CodeStyle, caption={Ecuación cuadrática en forma paralela, por Milton}, label={lst:cuadratic2.c}]{../../src/cuadratic2.c}

\lstinputlisting[language=C, style=CodeStyle, caption={Ecuación cuadrática con un hijo por proceso, por Milton}, label={lst:cuadratic3.c}]{../../src/cuadratic3.c}

\lstinputlisting[language=C, style=CodeStyle, caption={Ecuación cuadrática con memoria compartida, por Milton}, label={lst:cuadratic4.c}]{../../src/cuadratic4.c}

\lstinputlisting[language=C, style=CodeStyle, caption={Generador de números en formato binario}, label={lst:generador.c}]{../../src/generador.c}