\section{Procedimiento experimental}
Para el presente laboratorio fueron planteadas por el profesor de asignatura las siguientes tareas:
\begin{enumerate}
    \item Realizar un programa \textit{hijo\_padre\_abuelo} que muestre a través de sus ID la creación de ellos
    \item Ingresar por paso de parámetros tres valores enteros, indicar al padre que duplique el primer valor, al abuelo que eleve  a la potencia $3$ el segundo valor y que el hijo obtenga la raíz del tercer valor
    \item Dado un conjunto de valores del tipo ${a,b,c}$ obtener las raíces de una ecuación cuadrática cuyos factores son a,b,c. Resolver usando `fork()'
\end{enumerate}

Para la realización de las anteriores programas se hace uso del lenguaje \verb|C| que con librerías como \verb|sys/types.h| y \verb|unistd.h| se puede interactuar con el sistema operativo. En particular nos dan acceso a las siguientes herramientas:
\begin{enumerate}
    \item \verb|fork()| que permite duplicar un proceso.
    Esta función luego de generar el proceso hijo tiene un retorno para ambos procesos, el padre recibe por retorno el PID del hijo y el hijo recibe $0$ como retorno. Esto permite manejar la lógica para ambos procesos dentro del mismo código.
    \item \verb|pid_t| es el tipo de dato que representa el PID de un proceso.
    \item \verb|getpid()| que permite obtener el PID del proceso actual.
    \item \verb|getppid()| que permite obtener el PID del proceso padre.
    \item \verb|wait()| que permite esperar a que un proceso hijo termine su ejecución.
    Se puede pasar como parámetro el PID del proceso hijo a esperar, pero si se pasa \verb|NULL| se esperará hasta que cualquiera de los procesos hijos termine su ejecución.
    \item \verb|exec()| que permite ejecutar un programa en el sistema operativo.
    Esta función aunque no fué usada en la presente práctica es llamativa de mencionar puesto que permite convertir un proceso en otro, pudiendo reverenciarlo por su PID; Esta función es una manera más adecuada de ejecutar funciones de sistema en contraposición con \verb|system()| de la librería \verb|stdlib.h| puesto que, de acuerdo con \textcite{tanenbaum1997sistemas} puede conllevar a ciertos problemas de seguridad.
\end{enumerate}

Con todas las herramientas anteriores se puede proceder a realizar las tareas que se plantean en el presente laboratorio.